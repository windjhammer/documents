
\documentclass[12pt]{article}
\usepackage[utf8]{inputenc}
\usepackage[brazil]{babel}
\usepackage{geometry}
\usepackage{enumitem}
\usepackage{array}
\usepackage{titlesec}
\geometry{a4paper, margin=2.5cm}
\renewcommand{\arraystretch}{1.5}
\titleformat{\section}{\bfseries\Large\centering}{}{0em}{}

\begin{document}

% INÍCIO DOS CASOS DE TESTE

% Função auxiliar para gerar cada caso de teste
% Basta copiar a estrutura abaixo e mudar os valores

\newcommand{\casodeteste}[6]{
\begin{center}
 %   \textbf{\Large Caso de Teste}
\end{center}

\vspace{0.5cm}

\begin{tabular}{|>{\bfseries}p{4cm}|p{10cm}|}
    \hline
    ID & #1 \\
    \hline
    Nome & #2 \\
    \hline
    Ambiente & #3 \\
    \hline
    Pré-Condições & #4 \\
    \hline
    Procedimento & 
    \begin{itemize}[noitemsep,topsep=0pt]
        #5
    \end{itemize} \\
    \hline
    Pós-Condições & #6 \\
    \hline
\end{tabular}

\vspace{1cm}
}

% AQUI COMEÇAM OS CASOS

\casodeteste
{CT001}
{Verificar se a busca por palavras-chave retorna resultados relevantes.}
{Barra de busca}
{Acesso ao site G1 com conexão à internet.}
{
\item Acessar o site.
\item Digitar "educação" na barra de busca.
\item Pressionar Enter.
}
{Lista de notícias relacionadas ao tema educação.}

\casodeteste
{CT002}
{Verificar se a homepage carrega corretamente.}
{Página inicial}
{Acesso ao site.}
{
\item Acessar o site.
}
{Página inicial com notícias atualizadas e layout carregado.}

\casodeteste
{CT003}
{Testar a navegação entre categorias.}
{Menu principal}
{Página inicial carregada.}
{
\item Clicar em “Política”.
}
{Redirecionamento para a seção de política.}

\casodeteste
{CT004}
{Verificar a exibição de vídeos.}
{Player de vídeo}
{Conexão estável com internet.}
{
\item Clicar em uma notícia com vídeo.
\item Pressionar play.
}
{O vídeo deve iniciar sem travamentos.}

\casodeteste
{CT005}
{Validar o carregamento da versão mobile.}
{Responsividade}
{Acessar via dispositivo móvel.}
{
\item Acessar o site via celular.
}
{Layout adaptado para o tamanho da tela.}

\casodeteste
{CT006}
{Validar o funcionamento do botão de compartilhar.}
{Compartilhamento de notícia}
{Estar em uma notícia específica.}
{
\item Clicar no botão de compartilhar (WhatsApp, Facebook etc.).
}
{Exibição de opções de compartilhamento.}

\casodeteste
{CT007}
{Verificar o funcionamento da rolagem infinita.}
{Scroll automático}
{Estar em seção com rolagem.}
{
\item Rolar a página até o final.
}
{Novas notícias são carregadas automaticamente.}

\casodeteste
{CT008}
{Validar o tempo de resposta ao acessar uma notícia.}
{Tempo de carregamento}
{Internet estável.}
{
\item Clicar em uma notícia.
}
{Página carregada em menos de 3 segundos.}

\casodeteste
{CT009}
{Verificar a atualização de notícias.}
{Atualização em tempo real}
{Estar na página inicial por um tempo.}
{
\item Aguardar 10 minutos.
\item Atualizar a página.
}
{Exibição de novas notícias.}

\casodeteste
{CT010}
{Verificar erro ao digitar URL inexistente.}
{Tratamento de erro 404}
{Navegador aberto.}
{
\item Acessar https://g1.globo.com/aaaaa.
}
{Página de erro 404 personalizada do G1.}

\casodeteste
{CT011}
{Verificar se a navegação por notícias regionais funciona.}
{Menu de regiões}
{Página inicial aberta.}
{
\item Clicar em “São Paulo”.
}
{O site exibe apenas notícias relacionadas à região de São Paulo.}

\casodeteste
{CT012}
{Testar o funcionamento da função de zoom do navegador.}
{Acessibilidade visual}
{Estar na visualização padrão.}
{
\item Aplicar zoom (Ctrl + / Ctrl -).
}
{O layout permanece funcional e o conteúdo visível.}

\casodeteste
{CT013}
{Verificar se os links das redes sociais funcionam.}
{Ícones de redes sociais no rodapé}
{Rodapé visível.}
{
\item Clicar em “Facebook”.
}
{Abre a página oficial do G1 no Facebook.}

\casodeteste
{CT014}
{Detectar links quebrados.}
{Links internos}
{Página carregada.}
{
\item Clicar em 5 notícias aleatórias.
}
{Todos os links devem direcionar corretamente.}

\casodeteste
{CT015}
{Testar se notificações push são solicitadas.}
{Notificações}
{Acessar com navegador que permita notificações.}
{
\item Entrar no site e aguardar alguns segundos.
}
{Aparece solicitação para ativar notificações.}

\casodeteste
{CT016}
{Verificar se há link para a política de privacidade.}
{Rodapé do site}
{Página carregada.}
{
\item Rolar até o fim da página.
\item Clicar em “Política de Privacidade”.
}
{Usuário é redirecionado para a página com a política.}

\casodeteste
{CT017}
{Testar busca por data.}
{Filtro de data (se houver)}
{Página de busca aberta.}
{
\item Procurar notícias do dia 01/01/2024.
}
{Retornar somente notícias dessa data.}

\casodeteste
{CT018}
{Validar navegação por tags de temas.}
{Tags de artigos}
{Acessar uma notícia.}
{
\item Clicar em uma tag como “Educação”.
}
{Mostrar notícias relacionadas à tag.}

\casodeteste
{CT019}
{Verificar a exibição correta da seção de esportes.}
{Seção “Esporte”}
{Página inicial carregada.}
{
\item Clicar em “Esporte” no menu.
}
{Listagem de notícias esportivas atualizadas.}

\casodeteste
{CT020}
{Testar o carregamento de galerias de fotos.}
{Conteúdo multimídia}
{Galeria acessível.}
{
\item Clicar em uma galeria.
\item Navegar pelas imagens.
}
{Todas as imagens carregam corretamente.}

\casodeteste
{CT021}
{Verificar se a funcionalidade de comentários está ativa (caso exista).}
{Comentários de usuários}
{Estar logado (caso necessário).}
{
\item Rolar até a área de comentários de uma notícia.
}
{Permitir escrever e enviar comentários.}

\casodeteste
{CT022}
{Validar funcionamento do botão “voltar ao topo”.}
{Navegação facilitada}
{Página longa.}
{
\item Rolar até o final da página.
\item Clicar no botão “↑”.
}
{A página sobe automaticamente até o topo.}

\casodeteste
{CT023}
{Verificar a integração com login de serviços da Globo (Globoplay/G1 Play).}
{Autenticação}
{Possuir conta ativa.}
{
\item Clicar em “Entrar”.
\item Informar e-mail e senha.
}
{O login é realizado com sucesso e o nome do usuário aparece no topo.}

\casodeteste
{CT024}
{Testar busca por nome de jornalista.}
{Campo de busca}
{Saber o nome de um jornalista.}
{
\item Digitar “Renata Lo Prete” na busca.
}
{Notícias atribuídas à jornalista aparecem na busca.}

\casodeteste
{CT025}
{Validar se há formulário de feedback ou contato.}
{Contato/atendimento ao leitor}
{Página inicial carregada.}
{
\item Rolar até o rodapé.
\item Clicar em “Fale conosco” ou similar.
}
{Aparece uma página de contato ou formulário para envio de mensagem.}

% FIM DO DOCUMENTO
\end{document}
