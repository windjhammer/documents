
\documentclass[12pt]{article}
\usepackage[utf8]{inputenc}
\usepackage[brazil]{babel}
\usepackage{geometry}
\usepackage{enumitem}
\usepackage{array}
\geometry{a4paper, margin=2.5cm}
\title{Projeto Teste de Software - G1}
\author{Allyson Rubens, Breno Fernandes, João Vittor, Luiz Felipe, Tulio Montenegro\\
Faculdade UNINASSAU Olinda}
\date{2025}

\begin{document}

\maketitle

\section*{Introdução}

Este projeto tem como principal objetivo a aplicação prática dos conceitos teóricos estudados na disciplina de Teste de Software, por meio do desenvolvimento e execução de casos de teste aplicados a um sistema web real. Para essa atividade, o site escolhido foi o portal de notícias G1 (\texttt{https://g1.globo.com}), um dos maiores e mais acessados portais jornalísticos do Brasil, pertencente ao Grupo Globo.

A escolha do G1 não foi aleatória, sendo motivada por sua relevância no cenário nacional e pela variedade de funcionalidades oferecidas, que vão desde a exibição de notícias em tempo real até recursos multimídia, como vídeos, reportagens especiais, buscas temáticas e organização por editorias.

O portal G1 destaca-se por sua estrutura intuitiva e design responsivo, o que o torna acessível a diferentes tipos de dispositivos e perfis de usuários. Além disso, seu compromisso com a atualização constante e a confiabilidade das informações publicadas faz dele um excelente candidato para a realização de testes que envolvam critérios como usabilidade, performance e segurança.

A análise de um sistema com tais características permite não apenas a criação de casos de teste robustos, como também proporciona uma compreensão mais ampla da importância de garantir a qualidade de sistemas voltados para um grande volume de acessos simultâneos.

No contexto deste trabalho, buscou-se simular situações reais de uso, com o intuito de validar o funcionamento adequado das principais funcionalidades do site. Foram analisados elementos como o comportamento da barra de busca, a navegação por categorias, a reprodução de vídeos, a atualização de notícias em tempo real e a experiência do usuário de forma geral. Cada integrante do grupo ficou responsável por desenvolver cinco casos de teste, totalizando 25 cenários distintos.

Esses testes foram documentados seguindo um modelo estruturado que contempla campos como identificação, objetivo, pré-condições, passos para execução e o resultado esperado.

\section*{Casos de Teste}
\begin{center}
    \textbf{\Large Caso de Teste}
\end{center}

\vspace{0.5cm}

\renewcommand{\arraystretch}{1.5}
\begin{tabular}{|>{\bfseries}p{4cm}|p{10cm}|}
    \hline
    ID & CT001 \\
    \hline
    Nome & Verificar se a busca por palavras-chave retorna resultados relevantes. \\
    \hline
    Ambiente & Barra de busca. \\
    \hline
    Pré-Condições & Acesso ao site G1 com conexão à internet. \\
    \hline
    Procedimento & 
    \begin{itemize}
\item Acessar o site.
\item Digitar "educação" na barra de busca.
\item Pressionar Enter.
\item Resultado Esperado: Lista de notícias relacionadas ao tema educação.
\item 2. ID: CT002
\item Objetivo: Verificar se a homepage carrega corretamente.
\item Funcionalidade Testada: Página inicial.
\item Pré-condições: Acesso ao site.
\item Passos:
\item 1. Acessar o site.
\item Resultado Esperado: Página inicial com notícias atualizadas e layout carregado.
\item 3. ID: CT003
\item Objetivo: Testar a navegação entre categorias.
\item Funcionalidade Testada: Menu principal.
\item Pré-condições: Página inicial carregada.
\item Passos:
\item 1. Clicar em “Política”.
\item Resultado Esperado: Redirecionamento para a seção de política.
\item 4. ID: CT004
\item Objetivo: Verificar a exibição de vídeos.
\item Funcionalidade Testada: Player de vídeo.
\item Pré-condições: Conexão estável com internet.
\item Passos:
\item Clicar em uma notícia com vídeo.
\item Pressionar play.
\item Resultado Esperado: O vídeo deve iniciar sem travamentos.
\item 5. ID: CT005
\item Objetivo: Validar o carregamento da versão mobile.
\item Funcionalidade Testada: Responsividade.
\item Pré-condições: Acessar via dispositivo móvel.
\item Passos:
\item 1. Acessar o site via celular.
\item Resultado Esperado: Layout adaptado para o tamanho da tela.
\item 6. ID: CT006
\item Objetivo: Validar o funcionamento do botão de compartilhar.
\item Funcionalidade Testada: Compartilhamento de notícia.
\item Pré-condições: Estar em uma notícia específica.
\item Passos:
\item 1. Clicar no botão de compartilhar (WhatsApp, Facebook etc.).
\item Resultado Esperado: Exibição de opções de compartilhamento.
\item 7. ID: CT007
\item Objetivo: Verificar o funcionamento da rolagem infinita.
\item Funcionalidade Testada: Scroll automático.
\item Pré-condições: Estar em seção com rolagem.
\item Passos:
\item 1. Rolar a página até o final.
\item Resultado Esperado: Novas notícias são carregadas automaticamente.
\item 8. ID: CT008
\item Objetivo: Validar o tempo de resposta ao acessar uma notícia.
\item Funcionalidade Testada: Tempo de carregamento.
\item Pré-condições: Internet estável.
\item Passos:
\item 1. Clicar em uma notícia.
\item Resultado Esperado: Página carregada em menos de 3 segundos.
\item 9. ID: CT009
\item Objetivo: Verificar a atualização de notícias.
\item Funcionalidade Testada: Atualização em tempo real.
\item Pré-condições: Estar na página inicial por um tempo.
\item Passos:
\item Aguardar 10 minutos.
\item Atualizar a página.
\item Resultado Esperado: Exibição de novas notícias.
\item 10. ID: CT010
\item Objetivo: Verificar erro ao digitar URL inexistente.
\item Funcionalidade Testada: Tratamento de erro 404.
\item Pré-condições: Navegador aberto.
\item Passos:
\item 1. Acessar https://g1.globo.com/aaaaa.
\item Resultado Esperado: Página de erro 404 personalizada do G1.
\item 11. ID: CT011
\item Objetivo: Verificar se a navegação por notícias regionais funciona.
\item Funcionalidade Testada: Menu de regiões.
\item Pré-condições: Página inicial aberta.
\item Passos:
\item 1. Clicar em “São Paulo”.
    \end{itemize} \\
    \hline
    Pós-Condições & O site exibe apenas notícias relacionadas à região de São Paulo. 12. \\
    \hline
\end{tabular}

\vspace{1cm}

\begin{center}
    \textbf{\Large Caso de Teste}
\end{center}

\vspace{0.5cm}

\renewcommand{\arraystretch}{1.5}
\begin{tabular}{|>{\bfseries}p{4cm}|p{10cm}|}
    \hline
    ID & CT012 \\
    \hline
    Nome & Testar o funcionamento da função de zoom do navegador. \\
    \hline
    Ambiente & Acessibilidade visual. \\
    \hline
    Pré-Condições & Estar na visualização padrão. \\
    \hline
    Procedimento & 
    \begin{itemize}
\item 1. Aplicar zoom (Ctrl + / Ctrl -).
\item Resultado Esperado: O layout permanece funcional e o conteúdo visível.
\item 13. ID: CT013
\item Objetivo: Verificar se os links das redes sociais funcionam.
\item Funcionalidade Testada: Ícones de redes sociais no rodapé.
\item Pré-condições: Rodapé visível.
\item Passos:
\item 1. Clicar em “Facebook”.
\item Resultado Esperado: Abre a página oficial do G1 no Facebook.
\item 14. ID: CT014
\item Objetivo: Detectar links quebrados.
\item Funcionalidade Testada: Links internos.
\item Pré-condições: Página carregada.
\item Passos:
\item 1. Clicar em 5 notícias aleatórias.
\item Resultado Esperado: Todos os links devem direcionar corretamente.
\item 15. ID: CT015
\item Objetivo: Testar se notificações push são solicitadas.
\item Funcionalidade Testada: Notificações.
\item Pré-condições: Acessar com navegador que permita notificações.
\item Passos:
\item 1. Entrar no site e aguardar alguns segundos.
\item Resultado Esperado: Aparece solicitação para ativar notificações.
\item 16. ID: CT016
\item Objetivo: Verificar se há link para a política de privacidade.
\item Funcionalidade Testada: Rodapé do site.
\item Pré-condições: Página carregada.
\item Passos:
\item Rolar até o fim da página.
\item Clicar em “Política de Privacidade”.
\item Resultado Esperado: Usuário é redirecionado para a página com a política.
\item 17. ID: CT017
\item Objetivo: Testar busca por data.
\item Funcionalidade Testada: Filtro de data (se houver).
\item Pré-condições: Página de busca aberta.
\item Passos:
\item 1. Procurar notícias do dia 01/01/2024.
\item Resultado Esperado: Retornar somente notícias dessa data.
\item 18. ID: CT018
\item Objetivo: Validar navegação por tags de temas.
\item Funcionalidade Testada: Tags de artigos.
\item Pré-condições: Acessar uma notícia.
\item Passos:
\item 1. Clicar em uma tag como “Educação”.
\item Resultado Esperado: Mostrar notícias relacionadas à tag.
\item 19. ID: CT019
\item Objetivo: Verificar a exibição correta da seção de esportes.
\item Funcionalidade Testada: Seção “Esporte”.
\item Pré-condições: Página inicial carregada.
\item Passos:
\item 1. Clicar em “Esporte” no menu.
\item Resultado Esperado: Listagem de notícias esportivas atualizadas.
\item 20. ID: CT020
\item Objetivo: Testar o carregamento de galerias de fotos.
\item Funcionalidade Testada: Conteúdo multimídia.
\item Pré-condições: Galeria acessível.
\item Passos:
\item Clicar em uma galeria.
\item Navegar pelas imagens.
\item Resultado Esperado: Todas as imagens carregam corretamente.
\item 21. ID: CT021
\item Objetivo: Verificar se a funcionalidade de comentários está ativa (caso exista).
\item Funcionalidade Testada: Comentários de usuários.
\item Pré-condições: Estar logado (caso necessário).
\item Passos:
\item 1. Rolar até a área de comentários de uma notícia.
\item Resultado Esperado: Permitir escrever e enviar comentários.
\item 22. ID: CT022
\item Objetivo: Validar funcionamento do botão “voltar ao topo”.
\item Funcionalidade Testada: Navegação facilitada.
\item Pré-condições: Página longa.
\item Passos:
\item Rolar até o final da página.
\item Clicar no botão “↑”.
\item Resultado Esperado: A página sobe automaticamente até o topo.
\item 23. ID: CT023
\item Objetivo: Verificar a integração com login de serviços da Globo (Globoplay/G1 Play).
\item Funcionalidade Testada: Autenticação.
\item Pré-condições: Possuir conta ativa.
\item Passos:
\item Clicar em “Entrar”.
\item Informar e-mail e senha.
    \end{itemize} \\
    \hline
    Pós-Condições & O login é realizado com sucesso e o nome do usuário aparece no topo. 24. \\
    \hline
\end{tabular}

\vspace{1cm}

\begin{center}
    \textbf{\Large Caso de Teste}
\end{center}

\vspace{0.5cm}

\renewcommand{\arraystretch}{1.5}
\begin{tabular}{|>{\bfseries}p{4cm}|p{10cm}|}
    \hline
    ID & CT024 \\
    \hline
    Nome & Testar busca por nome de jornalista. \\
    \hline
    Ambiente & Campo de busca. \\
    \hline
    Pré-Condições & Saber o nome de um jornalista. \\
    \hline
    Procedimento & 
    \begin{itemize}
\item 1. Digitar “Renata Lo Prete” na busca.
    \end{itemize} \\
    \hline
    Pós-Condições & Notícias atribuídas à jornalista aparecem na busca. 25. \\
    \hline
\end{tabular}

\vspace{1cm}
\end{document}