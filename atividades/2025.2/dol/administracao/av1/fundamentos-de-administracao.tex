\documentclass[12pt]{article}
\usepackage[utf8]{inputenc}
\usepackage{geometry}
\usepackage{setspace}
\usepackage[brazil]{babel}
\geometry{a4paper, margin=1in}
\onehalfspacing

\title{Aplicação de Escolas da Administração na Connect LTDA}
\author{Breno Fernandes - 01169313}
\date{\today}

\begin{document}

\maketitle

A Connect LTDA enfrenta desafios na comunicação interna e na integração entre setores, o que compromete a execução de projetos e a satisfação dos clientes. Para lidar com esses problemas, é possível aplicar a Escola Clássica da Administração e a Escola das Relações Humanas.

A Escola Clássica, fundada por Henri Fayol, enfatiza a necessidade de uma estrutura organizacional bem definida, com hierarquias claras e divisão de tarefas. Aplicando essa teoria, a Connect poderia reestruturar seu organograma, definindo com clareza quem são os responsáveis por cada área e os fluxos de comunicação oficiais entre os setores. A empresa também pode adotar práticas como reuniões semanais de status, onde líderes de equipe atualizam uns aos outros sobre o andamento dos projetos. Isso evitaria retrabalhos e melhoraria a coordenação entre times presenciais e remotos, garantindo que todos tenham clareza sobre suas funções e prazos.

Já a Escola das Relações Humanas, proposta por Elton Mayo, traz uma abordagem focada no bem-estar dos colaboradores e na importância dos fatores sociais para a produtividade. Para a Connect, isso poderia ser implementado por meio de ações que fortaleçam a interação humana e a confiança entre equipes. Um exemplo seria criar eventos virtuais e presenciais de integração, fomentar rodas de feedback aberto e estabelecer programas de reconhecimento de boas práticas de comunicação. Além disso, investir em canais de comunicação interna menos formais, como grupos de bate-papo e plataformas colaborativas, pode estimular a participação ativa dos funcionários.

Com a aplicação combinada dessas duas escolas, a empresa pode tornar sua comunicação mais estruturada e, ao mesmo tempo, mais humana, fortalecendo a integração entre equipes, reduzindo falhas e atrasos e aumentando sua eficiência organizacional.

\end{document}

