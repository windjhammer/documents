\documentclass{article}
\usepackage[utf8]{inputenc}
\usepackage{enumitem}
\usepackage{geometry}
\usepackage[brazil]{babel}
\geometry{a4paper, margin=1in}

\begin{document}

\title{Atividade Contextualizada}
\author{Breno Fernandes - 01169313}
\date{\today}
\maketitle

\section*{Introdução}
falta de digitalização entre pequenos e médios produtores rurais. Esse público, essencial para a economia do país, enfrenta dificuldades no acesso a tecnologias de gestão, o que impacta diretamente na produtividade e competitividade no mercado. Ao observar essas dificuldades, surgiu a ideia de criar uma solução tecnológica simples, acessível e que agregasse valor ao dia a dia dos produtores, ajudando-os a melhorar suas operações e ampliar suas vendas de forma eficiente.

\section*{Definição do Problema}
\begin{enumerate}[label=\arabic*.]
    \item O problema existe pela falta de acesso de pequenos produtores rurais a tecnologias de gestão e vendas.
    \item Sofrem pequenos e médios produtores agrícolas, que perdem competitividade.
    \item O problema se faz presente na safra, na comercialização e na gestão financeira.
\end{enumerate}

\section*{Análise do Mercado}
\begin{enumerate}[label=\arabic*.]
    \item O mercado do agronegócio movimenta mais de R\$ 2 trilhões por ano no Brasil.
    \item Em 6 meses, é possível atingir 0,5\% do mercado de pequenos produtores de uma região.
    \item Existem cases como Agrotools e Agronow que atuam no suporte tecnológico ao campo.
\end{enumerate}

\section*{Modelo de Negócio}
\begin{enumerate}[label=\arabic*.]
    \item A proposição de valor é democratizar o acesso à gestão inteligente de produção e vendas.
    \item O principal segmento de clientes são pequenos produtores agrícolas e cooperativas.
    \item A receita virá da venda de licenças de uso da plataforma e serviços premium.
    \item Os canais serão marketing digital, parcerias com sindicatos rurais e eventos agrícolas.
\end{enumerate}

\section*{Funcionamento da Solução}
\begin{enumerate}[label=\arabic*.]
    \item O cliente usa um app para controlar produção, estoque e conectar-se a compradores.
    \item A solução automatiza processos, reduz perdas e melhora a renda percebida pelo cliente.
    \item O grau de inovação é médio-alto, integrando tecnologias simples a uma necessidade urgente.
\end{enumerate}

\section*{Projeção de Desenvolvimento}
\begin{enumerate}[label=\arabic*.]
    \item A meta nos primeiros 6 meses é captar 300 clientes ativos.
    \item Após 6 meses, crescer 50\% no número de clientes a cada semestre.
    \item Indicadores: número de usuários ativos, receita mensal e churn rate.
    \item As metas são viáveis devido à alta demanda reprimida e pouca concorrência local.
    \item Buscamos parceiros como Embrapa, Sebrae e investidores-anjo focados em agtech.
\end{enumerate}

\end{document}

