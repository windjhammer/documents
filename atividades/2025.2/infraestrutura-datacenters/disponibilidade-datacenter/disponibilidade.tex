\documentclass[a4paper]{article}
\usepackage{amsmath}
\usepackage[brazil]{babel}
\usepackage{datetime2}
\usepackage{pgfplots}
\usepackage{pgfplotstable}
\usepackage{lscape}
\usepackage{geometry}
\pgfplotsset{compat=1.18}

\begin{document}

\title{Cálculo de Disponibilidade do Sistema}
\author{Breno Fernandes - 01169313}
\date{\today}
\maketitle

\section{Resultados}

Com base no gráfico, temos os seguintes cálculos:

\begin{itemize}
    \item \textbf{MTBF}: 4 dias
    \item \textbf{MTTR}: 3 dias
    \item \textbf{Disponibilidade do sistema}: 
    \begin{align*}
        D &= \frac{MTBF}{MTBF + MTTR} \times 100\\
        &= \frac{4}{4 + 3} \times 100\\
        &= \frac{4}{7} \times 100\\
        &\approx 57,14\%
    \end{align*}
\end{itemize}

Portanto, a disponibilidade do sistema nesta semana foi de aproximadamente \textbf{57,14\%}.

\pgfplotsset{compat=1.18}

\begin{tikzpicture}
\begin{axis}[
    width=14cm,
    height=6cm,
    xlabel=Dias,
    ylabel=Disponibilidade,
    xmin=1,
    xmax=7,
    ymin=-0.2,
    ymax=1.2,
    xtick={1,2,...,7},
    xticklabels={Dia 1, Dia 2, Dia 3, Dia 4, Dia 5, Dia 6, Dia 7},
    ytick={0,1},
    yticklabels={Indisponível, Disponível},
    grid=both,
    grid style={dashed,gray!30},
    title={Disponibilidade do Host - Últimos 7 Dias},
    axis lines=left,
    enlargelimits=false
]

\draw[blue, thick] (1,1) -- (5.5,1);    
\draw[blue, thick] (6.375,1) -- (7,1);  

\draw[red, thick] (5.5,0) -- (6.375,0); 

\draw[blue, thick] (5.5,1) -- (5.5,0);    
\draw[blue, thick] (6.375,0) -- (6.375,1);  

\end{axis}
\end{tikzpicture}

\section{Disponibilidade de um Data Center}

\textbf{2 – O que significa disponibilidade de um Data Center?}  \\
Disponibilidade de um Data Center refere-se ao tempo em que ele permanece operacional e acessível com os serviços rodando sem interrupções ou falhas.

\textbf{3 – De qual infraestrutura um Data Center depende para a sua disponibilidade?}  \\
A disponibilidade de um Data Center depende de diversos fatores, incluindo energia elétrica (tanto da concessionária quanto de nobreaks e geradores), climatização, rede, segurança, controle de acesso, além de hardware e software confiáveis.

\textbf{4 – Cálculo do percentual de disponibilidade de um Data Center}  \\
Deve-se considerar o espaçdo de um ano com 365 dias. Se o tempo total de indisponibilidade foi de 2 dias, então:
\begin{align*}
    D &= \left( \frac{365 - 2}{365} \right) \times 100 \\
    &= \left( \frac{363}{365} \right) \times 100 \\
    &\approx 99,45\%
\end{align*}
Portanto, a disponibilidade do Data Center foi aproximadamente \textbf{99,45\%}.

\textbf{5 – Exemplo de sistema com redundância}  \\
Um exemplo é um servidor de aplicação configurado em \textbf{cluster} proxmox com balanceamento de carga e alta disponibilidade, é configurado um sistema de armazenamento CEPH ou JBOD para permitir que todos os dados estejam sempre presentes e acessíveis a todos os servidores do cluster a todo momento. Caso um servidor falhe, outro assume automaticamente, garantindo continuidade no serviço.

\textbf{6 – Cálculo de módulos de potência para nobreak}  \\
A configuração \textbf{N+2} significa que, além dos módulos necessários (N), são adicionados dois módulos extras para redundância.  
Cálculo:
\begin{align*}
    N &= \frac{100}{16} = 6,25
\end{align*}
Arredondando para cima: \textbf{7 módulos} são necessários para suprir a carga.  
Com a redundância \textbf{N+2}, a configuração total será:
\begin{align*}
    7 + 2 = 9 \text{ módulos de 16 kW}
\end{align*}

\end{document}

