\documentclass{article}
\usepackage[utf8]{inputenc}
\usepackage[T1]{fontenc}
\usepackage[brazil]{babel}
\usepackage{geometry}
\usepackage{enumitem}
\usepackage{hyperref}
\usepackage{xcolor}
%\usepackage{fontawesome5}
\usepackage{enumitem}
\usepackage{xcolor}

\geometry{a4paper, margin=2cm}

\title{Dois Casos de Data Centers: Empresarial vs. Educacional}
\author{Breno Fernandes - 01169313}
\date{\today}

\begin{document}

\maketitle

\section*{Caso 1: A Scala Data Centers - Gigante das Nuvens}

\subsection*{O que eles fazem?}
A \textbf{Scala Data Centers} é a \textit{gigante dos data centers hyperscale} na América Latina, especializada em infraestruturas de alta densidade para big techs como \textbf{Amazon, Microsoft e Google}. Fundada em 2020 pela DigitalBridge, a empresa já opera \textbf{12 data centers} (10 no Brasil) e tem \textbf{11 em construção}, com projetos que ultrapassam \textbf{R\$ 40 bilhões em investimentos}.

\begin{itemize}[leftmargin=*,noitemsep]
    \item \textbf{Foco estratégico:}
    \begin{itemize}[leftmargin=*,topsep=0pt]
        \item \textit{Hyperscale}: Projetam data centers com capacidade energética de até \textbf{600~MW} (equivalente ao consumo de Brasília), como o \textbf{Campus Tamboré (SP)}, maior complexo da região
        \item \textit{Sustentabilidade radical}: Operam com \textbf{100\% energia renovável}, \textbf{PUE mais baixo da AL} (1,3) e neutralidade de carbono desde 2021
        \item \textit{Inovação em IA}: Distrito \textbf{``Scala AI City'' no RS} para treinamento de modelos de IA (R\$ 3~bi iniciais)
    \end{itemize}
    
    \item \textbf{Expansão agressiva:}
    \begin{itemize}[leftmargin=*,topsep=0pt]
        \item Presença em \textbf{Brasil, Chile, Colômbia e México}, com novos projetos em \textbf{Fortaleza (CE)} e \textbf{Eldorado do Sul (RS)}
        \item Parceria com a \textbf{Vertiv} para refrigeração adiabática (130~MW de capacidade)
        \item Metodologia \textit{FastDeploy}, redução de 50\% no tempo de construção
    \end{itemize}
    
    \item \textbf{Diferenciais competitivos:}
    \begin{itemize}[leftmargin=*,topsep=0pt]
        \item Contratos flexíveis para \textit{hyperscalers} com infraestrutura modular
        \item \textbf{ESG como DNA}: 2.900~GWh de energia verde até 2033 e bolsas para comunidades
    \end{itemize}
\end{itemize}

\noindent
Em resumo, a Scala é a \textbf{arquiteta da infraestrutura digital latino-americana}, combinando escala industrial, inovação disruptiva e compromisso ambiental.
\subsection*{Como funciona?}
\begin{itemize}[leftmargin=*]
    \item \textbf{Tamanho GG}: Operam data centers \textbf{hyperscale} (aqueles que parecem cidades de tão grandes)
    \item \textbf{Energia Verde}: Usam 100\% de energia renovável - até vento e sol viram eletricidade!
    \item \textbf{Cooler Sustentável}: Sistema de resfriamento inteligente que gasta menos energia que ar-condicionado comum
\end{itemize}

\subsection*{Projetos Destaque}
\begin{itemize}
    \item \textbf{Campus Tamboré (SP)}: Vai ser maior que muitos estádios de futebol - capacidade pra 450MW (isso dá pra abastecer Brasília!)
    \item \textbf{Data Center Vertical (Barueri)}: Prédio de vários andares só com servidores - parece um arranha-céu de tecnologia
\end{itemize}

\subsection*{Por que eles se destacam?}
\begin{itemize}
    \item Constroem data centers em 10 meses (mais rápido que obra de rodovia!)
    \item Parceria com cabos submarinos internacionais
    \item Já têm clientes famosos tipo AWS e Microsoft
\end{itemize}

\section*{Caso 2: A Faculdade Multicampi - Qual Data Center Escolher?}

\subsection*{O problema}
Imagine uma organização educacional com diversos campi. Cada campus possui uma pequena infraestrutura de TI com acesso à Internet e todos os sistemas de informação desta organização são na Web. Imagine que você foi contratado por essa organização para administrar a infraestrutura de TI. Tendo em vista os altos custos de implantação de um EDC, você poderia recomendar o uso de IDC ou de CDC para esta organização?
\subsection*{Minha recomendação: Nuvem (IDC) é o caminho!}
\begin{itemize}
    \item \textbf{Porque não CDC?} Colocation exigiria comprar servidores caros pra cada campus - tipo ter que montar uma lan house em cada unidade
    \item \textbf{Vantagens da Nuvem}:
    \begin{itemize}
        \item Paga só o que usar (igual conta de luz)
        \item Acesso de qualquer lugar - até do celular do professor
        \item Atualizações automáticas (sem precisar de "cara do TI" correndo entre os campi)
    \end{itemize}
\end{itemize}

\subsection*{Na prática...}
\begin{itemize}
    \item \textbf{Exemplo 1}: Usar Google Classroom + AWS Educate (já tem desconto pra ensino)
    \item \textbf{Exemplo 2}: Migrar o sistema de matrículas pra Microsoft Azure (que já cumpre leis de proteção de dados)
\end{itemize}

\subsection*{E se vier um pico de acesso?}
A nuvem escala automaticamente! Se todo mundo tentar ver as notas ao mesmo tempo, o sistema não cai - diferente de um servidor local que ia virar uma fritura.

\end{document}
