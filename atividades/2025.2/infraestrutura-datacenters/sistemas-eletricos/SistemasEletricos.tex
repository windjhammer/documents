\documentclass[12pt]{article}
\usepackage[utf8]{inputenc}
\usepackage[brazil]{babel}
\usepackage[T1]{fontenc}
\usepackage{lmodern}
\usepackage[a4paper, margin=1in]{geometry}
\usepackage{graphicx}
\usepackage{hyperref}
\usepackage{amsmath}
\usepackage{tabularx}
\usepackage{booktabs}
\usepackage{geometry}
\usepackage{ragged2e}
\usepackage{pgfplots}
\usepackage{geometry}
\pgfplotsset{compat=1.18}
\usepgfplotslibrary{dateplot}


\title{Sistemas elétricos de datacenters}
\author{Breno Fernandes - 01169313}
\date{\today}

\begin{document}

\maketitle

\section*{Configuração do Datacenter}
O servidor escolhido é um \textbf{Dell PowerEdge R550}. Segundo o documento, serão utilizados 45 servidores distribuídos em 3 racs, portanto, cada rack comportará 15 servidores.
Segundo o \textit{datasheet} com a descrição das configurações do servidor, existem versões com fonte de 600w e 800w, optei por seguir com a configuração de 800w para a realização dos cálculos.

\renewcommand{\arraystretch}{1.3}
\begin{table}[h]
    \centering
    \begin{tabular}{|c|l|c|}
        \hline
        \textbf{Equipamento} & \textbf{Descrição} & \textbf{Carga (kW)} \\
        \hline
        RACK01 & Rack 42U - 15x Servidores Dell R550 & 12,0 \\
        \hline
        RACK02 & Rack 42U - 15x Servidores Dell R550 & 12,0 \\
        \hline
        RACK03 & Rack 42U - 15x Servidores Dell R550 & 12,0 \\
        \hline
        \multicolumn{2}{|l|}{\textbf{Carga total do datacenter}} & \textbf{36,0} \\
        \hline
    \end{tabular}
    \caption{Distribuição da carga de servidores Dell R550}
\end{table}

\section*{Cálculo do fator de crescimento}
A potência inicial do datacenter é \textbf{36.0 kW} (ano 0). Considerando um crescimento anual de \textbf{10\%} a evolução é calculada pela fórmula:

\[
P_{\text{final}} = P_{\text{inicial}} \times (1 + \text{taxa})^{\text{anos}}
\]

\noindent onde:
\begin{itemize}
    \item \( P_{\text{inicial}} = 36.0 \, \text{kW} \)
    \item \( \text{taxa} = 10\% = 0.10 \)
    \item \( \text{anos} = 5 \)
\end{itemize}

\newpage
\vspace*{-\baselineskip}
\subsection*{Crescimento Anual}
\begin{table}[h]
    \centering
    \small 
    \setlength{\tabcolsep}{8pt} 
    \begin{tabular}{|c|c|r|}
        \hline
        \textbf{Ano} & \textbf{Fator de Crescimento} & \textbf{Potência (kW)} \\
        \hline
        0 & \( 1.10^0 = 1.00 \) & 36.00 \\
        \hline
        1 & \( 1.10^1 = 1.10 \) & 36.00 \(\times\) 1.10 = 39.60 \\
        \hline
        2 & \( 1.10^2 = 1.21 \) & 39.60 \(\times\) 1.10 = 43.56 \\
        \hline
        3 & \( 1.10^3 = 1.33 \) & 43.56 \(\times\) 1.10 = 47.92 \\
        \hline
        4 & \( 1.10^4 = 1.46 \) & 47.92 \(\times\) 1.10 = 52.71 \\
        \hline
        5 & \( 1.10^5 = 1.61 \) & 52.71 \(\times\) 1.10 = \boxed{57.98} \\
        \hline
    \end{tabular}
    \caption{Cálculo detalhado da potência com crescimento anual de 10\%.}
\end{table}

Após 5 anos, a potência total projetada é:
\[
\boxed{57.98 \, \text{kW}} \quad (\text{aumento de 61\% sobre a carga inicial}).
\]
\section*{Cargas para Espaço de Suporte}
\subsection*{Ajuste}
Um aumento adicional de \textbf{15\%} é aplicado à potência total para considerar:  
\begin{itemize}
    \item NOC 
    \item Entrada de serviços de telecomunicações
    \item Armazenagem de equipamentos e suprimentos
\end{itemize}

\[
P_{\text{final\_ajustada}} = P_{\text{final}} \times (1 + 0.15)
\]

\subsection*{Tabela de Cálculo}
\begin{table}[h!]
    \centering
    \small
    \setlength{\tabcolsep}{8pt}
    \begin{tabular}{|c|c|r|}
        \hline
        \textbf{Fator} & \textbf{Descrição} & \textbf{Potência (kW)} \\
        \hline
        Potência após 5 anos & Base & 57.98 \\
        \hline
        Aumento de 15\% & \( 57.98 \times 1.15 \) & \boxed{66.68} \\
        \hline
    \end{tabular}
    \caption{Ajuste para cargas de suporte do datacenter.}
\end{table}

\subsection*{Resultado Ajustado}
A potência total, incluindo infraestrutura de suporte, é:
\[
\boxed{66.68 \, \text{kW}} \quad (\text{aumento total de 85.2\% sobre a carga inicial}).
\]

\subsection*{Carga de Iluminação}
\begin{itemize}
    \item Densidade: \( 25 \, \text{W/m}^2 \)
    \item Área: \( 15 \, \text{m}^2 \)
    \item Cálculo:
    \[
    25 \times 15 = 375 \, \text{W} = \boxed{0.38 \, \text{kW}}
    \]
    \item Total parcial:
    \[
    66.68 + 0.38 = \boxed{67.06 \, \text{kW}}
    \]
\end{itemize}

\subsection*{Carga do Sistema UPS (Nobreaks)}
\begin{itemize}
    \item Fator de bateria: \( 15\% \)
    \item Fator de ineficiência: \( 10\% \)
    \item Cálculo:
    \[
    67.06 \times (0.15 + 0.10) = 67.06 \times 0.25 = \boxed{16.77 \, \text{kW}}
    \]
    \item Total parcial:
    \[
    67.06 + 16.77 = \boxed{83.83 \, \text{kW}}
    \]
\end{itemize}

\subsection*{Carga Mecânica (Climatização)}
\begin{itemize}
    \item Fator de carga: \( 65\% \)
    \item Cálculo:
    \[
    83.83 \times 0.65 = \boxed{54.49 \, \text{kW}}
    \]
    \item Total parcial:
    \[
    83.83 + 54.49 = \boxed{138.32 \, \text{kW}}
    \]
\end{itemize}

\subsection*{Margem de Segurança}
\begin{itemize}
    \item Fator de segurança: \( 25\% \)
    \item Cálculo:
    \[
    138.32 \times 0.25 = \boxed{34.58 \, \text{kW}}
    \]
    \item Total final:
    \[
    138.32 + 34.58 = \boxed{172.90 \, \text{kW}}
    \]
\end{itemize}

\section*{Resumo das Cargas}
\begin{table}[h!]
    \centering
    \small
    \begin{tabular}{|l|c|r|}
        \hline
        \textbf{Componente} & \textbf{Fator} & \textbf{Potência (kW)} \\
        \hline
        Potência inicial do datacenter & - & 36.00 \\
        Crescimento 5 anos & \(1.10^5\) & +21.98 \\
        Cargas de suporte & 15\% & +8.70 \\
        Iluminação & \(25 \, \text{W/m}^2\) & +0.38 \\
        Sistema UPS & 25\% & +16.77 \\
        Carga mecânica & 65\% & +54.49 \\
        Margem de segurança & 25\% & +34.58 \\
        \hline
        \textbf{Total Final} & & \boxed{172.90} \\
        \hline
    \end{tabular}
    \caption{Composição detalhada da carga total do datacenter}
\end{table}

\section*{Resultado Final}
A potência total projetada, incluindo todos os fatores de carga e margens de segurança, é:
\[
\boxed{172.90 \, \text{kW}} \quad (\text{aumento de 380.3\% sobre a carga inicial}).
\]

\vspace{3mm}
\noindent \textit{Observação:} O cálculo do aumento total considera a relação:
\[
\left(\frac{172.90}{36.0} - 1\right) \times 100 = 380.3\% 
\]

\section*{Cálculo da Corrente do Alimentador Elétrico}

\subsection*{Fórmula}
A corrente do alimentador é determinada pela potência total e tensão trifásica:
\[
P_{\text{total}} = U_{\text{trifásico}} \cdot I \cdot \sqrt{3} \quad \Rightarrow \quad I = \frac{P_{\text{total}}}{U_{\text{trifásico}} \cdot \sqrt{3}}
\]

\begin{itemize}
    \item Potência total (\( P_{\text{total}} \)): \( 172,90 \, \text{kW} = 172.900 \, \text{W} \)
    \item Tensão trifásica padrão (\( U_{\text{trifásico}} \)): \( 380 \, \text{V} \)
    \item \( \sqrt{3} \approx 1,732 \)
\end{itemize}

\[
I = \frac{172.900}{380 \times 1,732} = \frac{172.900}{658,16} \approx \boxed{262,7 \, \text{A}}
\]

\subsection*{Tabela}
\begin{table}[h!]
    \centering
    \small
    \begin{tabular}{|l|c|c|}
        \hline
        \textbf{Parâmetro} & \textbf{Símbolo} & \textbf{Valor} \\
        \hline
        Potência total & \( P_{\text{total}} \) & 172,90 kW \\
        Tensão trifásica & \( U_{\text{trifásico}} \) & 380 V \\
        Fator \( \sqrt{3} \) & - & 1,732 \\
        Corrente calculada & \( I \) & \boxed{262,7 \, \text{A}} \\
        \hline
    \end{tabular}
    \caption{Parâmetros para cálculo da corrente do alimentador.}
\end{table}

\section*{Evolução da Potência}
\begin{figure}[h!]
    \centering
    \begin{tikzpicture}
        \begin{axis}[
            width=0.9\textwidth,
            height=0.7\textwidth,
            xlabel={Etapas},
            ylabel={Potência (kW)},
            symbolic x coords={Inicial, Crescimento, Suporte, Iluminação, UPS, Mecânica, Margem},
            xtick=data,
            x tick label style={
                rotate=45,
                anchor=east, 
                font=\small,
                align=center 
            },
            ymin=0,
            ymax=200,
            nodes near coords,
            nodes near coords style={
                font=\tiny,
                rotate=45,
                anchor=west
            },
            enlarge x limits=0.2,  
            grid=both,
            major grid style={line width=.2pt,draw=gray!50},
            ]
            \addplot[blue, thick, mark=*, mark options={fill=white}] coordinates {
                (Inicial, 36.00)
                (Crescimento, 57.98)
                (Suporte, 66.68)
                (Iluminação, 67.06)
                (UPS, 83.83)
                (Mecânica, 138.32)
                (Margem, 172.90)
            };
        \end{axis}
    \end{tikzpicture}
    \caption{Aumento da potência total após cada componente adicionado.}
\end{figure}

\section*{Considerações Finais}
\begin{itemize}
    \item \textbf{Aumento total:} A potência final de \boxed{172,90 \, \text{kW}} representa um aumento de \textbf{380,3\%} em relação à carga inicial (36 kW).
    \item \textbf{Componente crítico:} A carga mecânica climatização foi o maior incremento individual (+54,49 kW), reforçando a necessidade de planejamento robusto para sistemas dessa natureza.
    \item \textbf{Margem de segurança:} Os 25\% adicionais garantem flexibilidade para expansões não previstas ou picos de demanda.
    \item \textbf{Corrente do alimentador:} Para a potência final, a corrente estimada é de \boxed{262,7 \, \text{A}} (em 380 V trifásico), exigindo cabos de qualidade e dimensão adequada.
\end{itemize}


\end{document}
