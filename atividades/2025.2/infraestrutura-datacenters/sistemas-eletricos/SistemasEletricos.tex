\documentclass[12pt]{article}
\usepackage[utf8]{inputenc}
\usepackage[brazil]{babel}
\usepackage[T1]{fontenc}
\usepackage{lmodern}
\usepackage[a4paper, margin=1in]{geometry}
\usepackage{graphicx}
\usepackage{hyperref}
\usepackage{amsmath}
\usepackage{tabularx}
\usepackage{booktabs}
\usepackage{geometry}
\usepackage{ragged2e}

\title{Sistemas elétricos de datacenters}
\author{Breno Fernandes - 01169313}
\date{\today}

\begin{document}

\maketitle

\section*{Configuração do Datacenter}
O servidor escolhido é um \textbf{Dell PowerEdge R550}. Segundo o documento, serão utilizados 45 servidores distribuídos em 3 racs, portanto, cada rack comportará 15 servidores.
Segundo o \textit{datasheet} com a descrição das configurações do servidor, existem versões com fonte de 600w e 800w, optei por seguir com a configuração de 800w para a realização dos cálculos.

\renewcommand{\arraystretch}{1.3}
\begin{table}[h]
    \centering
    \begin{tabular}{|c|l|c|}
        \hline
        \textbf{Equipamento} & \textbf{Descrição} & \textbf{Carga (kW)} \\
        \hline
        RACK01 & Rack 42U - 15x Servidores Dell R550 & 12,0 \\
        \hline
        RACK02 & Rack 42U - 15x Servidores Dell R550 & 12,0 \\
        \hline
        RACK03 & Rack 42U - 15x Servidores Dell R550 & 12,0 \\
        \hline
        \multicolumn{2}{|l|}{\textbf{Carga total do datacenter}} & \textbf{36,0} \\
        \hline
    \end{tabular}
    \caption{Distribuição da carga de servidores Dell R550}
\end{table}

\section*{Cálculo do Fator de Crescimento}

O fator de crescimento é calculado utilizando a seguinte fórmula:

\[
  {Crescimento} = {Carga} \left(1 + \right.{taxa})^{tempo}
$$$$
\]
\[
36 \times (1 + (10/100))^5 = 57.97836
\]

Foi considerado um crescimento de 10\% num espaço de tempo de 5 anos, portanto:
\[
12.0 \times 1.61051 = 19.33 \ \mathrm{kW} \ (\text{por\ rack})
\]
\[
36.0 \times 1.61051 = 57.85 \ \mathrm{ kW (total)}
\]
Dessa forma, a carga total do projeto considerando o fator de crescimento será de \textbf{57,85} kW.

\end{document}
