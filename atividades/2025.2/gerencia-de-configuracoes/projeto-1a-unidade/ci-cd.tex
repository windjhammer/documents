\documentclass[a4paper,12pt]{article}
\usepackage[utf8]{inputenc}
\usepackage{graphicx}
\usepackage{fancyhdr}
\usepackage{lipsum}
\usepackage{listings}
\usepackage{color}
\definecolor{lightgray}{gray}{0.9}

\lstset{
    showstringspaces=false,
    basicstyle=\ttfamily,
    keywordstyle=\color{blue},
    commentstyle=\color[grey]{0.6},
    stringstyle=\color[RGB]{255,150,75}
}

\newcommand{\inlinecode}[2]{\colorbox{lightgray}{\lstinline[language=#1]$#2$}}

\begin{document}
\begin{titlepage}
    \centering
    \includegraphics[width=0.4\textwidth]{./media/uninassau-logo.png}\par\vspace{1cm}
    
    {\huge\bfseries Trabalho sobre Integração Contínua (CI) e Entrega Contínua (CD)\par}\vspace{1cm}
    
    \textbf{Disciplina:} Gerência de Configurações e Dependências\par
    \textbf{Professor:} Silvio Caetano\par
    \textbf{Alunos:} Breno Fernandes (01169313)\par
    
    \vfill
    
    {\large Olinda, Março de 2024\par}
\end{titlepage}

\pagestyle{fancy}
\fancyhf{}
\fancyfoot[L]{Uninassau}
\fancyhead[C]{Integração Contínua e Entrega Contínua \textbf{CI/CD}}
\fancyfoot[R]{\thepage}

\newpage

\section*{Introdução}
A Integração Contínua e Entrega Contínua são práticas fundamentais no desenvolvimento de software moderno. O úso dessas práticas visa garantir que mudanças no código sejam testadas, criadas, integradas e entregues de maneira automatizada, reduzindo erros e aumentando a confiabilidade e eficiência do sistema.  

Este trabalho tem como objetivo explorar os conceitos de CI/CD, apresentando suas diferenças, evidenciando as principais ferramentas, mostrando a importância do processo de automatizar o desenvolvimento de software e também evidenciando os desafios e boas práticas para sua implementação. Além disso, serão respondidas questões propostas tema demonstrando sua aplicação prática em um ambiente real.
\newpage{}
\section*{Desenvolvimento}

\subsection*{Diferença entre CI e CD}
O CI/CD são práticas essenciais no ciclo de vida do desenvolvimento de software, mas possuem propósitos distintos dentro do processo de automação.  

A \textbf{Integração Contínua } refere-se ao processo de integrar continuamente o código desenvolvido por diferentes membros da equipe em um repositório central. Cada nova alteração no código passa por testes automatizados para garantir que não haja erros antes de ser realizado o \textit{merge}. O objetivo principal do CI é detectar e corrigir problemas rapidamente, garantindo que o software esteja sempre em um estado funcional.  

Já a \textbf{Entrega Contínua} ocorre após a CI e se concentra em garantir que o software esteja sempre pronto para implantação. Isso significa que, após a integração do código e a execução dos testes, o sistema é preparado para ser entregue automaticamente em um ambiente de produção ou homologação. A CD reduz o tempo entre o desenvolvimento e a entrega ao usuário final, garantindo que novas funcionalidades sejam disponibilizadas de maneira rápida e confiável.  

\subsection*{Exemplos práticos}  
Um exemplo prático de \textbf{CI} seria um time de desenvolvedores trabalhando em um projeto compartilhado no GitHub. A cada novo commit feito no repositório, um pipeline do GitHub Actions ou Jenkins é acionado para rodar testes automatizados e verificar a qualidade do código. Se os testes passarem, o código pode ser integrado automaticamente ao branch principal.  

Por outro lado, um exemplo de \textbf{CD} seria uma aplicação hospedada na AWS que utiliza um pipeline do GitLab CI/CD para automatizar a implantação. Assim que o código passa nos testes da CI, ele é automaticamente empacotado e enviado para um ambiente de produção usando ferramentas como \textit{Docker} e \textit{Kubernetes}, permitindo que novas versões do software sejam disponibilizadas para os usuários sem intervenção manual.  

Ferramentas como \textit{Docker} e \textit{Kubernetes} desempenham um papel fundamental na fase de \textbf{CD} da esteira de \textbf{CI/CD}. O \textit{Docker} é utilizado para empacotar aplicações e suas dependências em contêineres, garantindo que o software funcione da mesma maneira em qualquer ambiente, seja no computador do desenvolvedor ou em produção. Já o \textit{Kubernetes} atua na fase de implantação e gerenciamento, sendo responsável por orquestrar os contêineres, garantindo escalabilidade, alta disponibilidade e automação no deploy das aplicações.  

Enquanto a CI foca na validação e testes de código, e usa ferramentas como \textit{Jenkins, CircleCI e Teamcity}, Docker e Kubernetes entram na fase de CD, facilitando a entrega automatizada e eficiente do software em diferentes ambientes de execução.
\subsection*{Ferramentas utilizadas}
Diferentes ferramentas desempenham papéis essenciais tanto na CI quanto na CD. Algumas das mais comuns são:  

- \textbf{Para CI:} \newline{} 
- \textit{Jenkins} – Uma das ferramentas mais populares de automação de integração, altamente configurável.  \newline{}
  -  \textit{GitHub Actions} – Integrado ao GitHub, permite definir workflows automatizados para testes.  \newline{}
  -  \textit{GitLab CI/CD} – Plataforma integrada ao GitLab para rodar testes e validar código automaticamente.  \newline{}
  -  \textit{CircleCI} – Focado em pipelines de CI rápidos e eficientes.  \newline{}
  -  \textit{Travis CI}– Comum em projetos open source, permite testar código em diferentes ambientes.  \newline{}

  - \textbf{Para CD:} \newline
- \textit{ArgoCD} – Gerencia deploys contínuos em ambientes Kubernetes. \newline 
- \textit{Spinnake} – Plataforma de entrega contínua usada por empresas como Netflix e Google. \newline
- \textit{AWS CodeDeploy} – Automação de implantação na infraestrutura da AWS. \newline
- \textit{Azure DevOps} – Ferramenta integrada para CI/CD em ambientes Microsoft. \newline
- \textit{Google Cloud Build} – Automatiza a entrega contínua em ambientes Google Cloud. \newline

Enquanto a \textbf{CI} garante que o código esteja sempre testado e pronto para uso, a \textbf{CD} foca em disponibilizar esse código para produção de forma automatizada e confiável. Juntas, essas práticas aumentam a eficiência do desenvolvimento de software e reduzem o risco de falhas em produção.  

\subsection*{Análise de Pipeline CI/CD}  
O pipeline de \textbf{CI/CD} apresentado é composto por três etapas principais: \textit{build, test e deploy}.  

A primeira etapa, \textit{build}, é responsável pela compilação do código. Durante essa fase, o comando \inlinecode{java}{mvn package} é executado, indicando o uso do Maven para empacotamento do projeto, gerando um arquivo \inlinecode{java}{.jar}.  

A segunda etapa, \textit{test}, executa os testes automatizados do projeto. O comando \inlinecode{java}{mvn test} roda os testes definidos na aplicação para garantir que o código funciona corretamente antes de avançar para as próximas fases. Caso algum teste falhe, o pipeline é interrompido e o código não segue para as próximas etapas.  

A terceira etapa, \textit{deploy}, realiza a implantação do arquivo gerado na fase de \textit{build}. O comando \inlinecode{java}{scp target/app.jar servidor:/app} copia o arquivo \inlinecode{java}{.jar} para um servidor remoto, permitindo que ele seja executado em produção.  

O deploy ocorre \textbf{apenas na branch} \textit{'main'}, conforme definido pelo comando declarativo \inlinecode{java}{only: - main}. Isso significa que mudanças em \textit{branches} de desenvolvimento ou \textit{branches} de features não serão implantadas automaticamente. Esse comportamento evita que versões instáveis ou em desenvolvimento sejam enviadas para produção, garantindo que apenas código revisado e aprovado na \textit{'main'} seja disponibilizado para os usuários.  

\subsection*{Estudos de Caso (Análise de Problemas)}  

\subsubsection*{Caso 1: Problemas de compatibilidade entre versões}  
A equipe enfrenta dificuldades devido à integração manual do código, o que frequentemente causa problemas de compatibilidade entre versões. Esse cenário pode ser resolvido com a implementação do CI.  

A CI automatiza o processo de integração de código, garantindo que todas as alterações feitas pelos desenvolvedores sejam testadas continuamente antes de serem mescladas ao repositório principal. Isso pode ser feito utilizando \textit{pipelines} automatizados que executam testes unitários e de integração sempre que um novo \textit{commit} é realizado.  

A solução mais adequada é configurar uma pipeline de CI utilizando, por exemplo, o GitHub Actions. Esse pipeline deveria seguir as seguintes etapas:  \newline
1. \textbf{Build automático} – Compilar o código a cada commit para identificar problemas estruturais.  \newline
2. \textbf{Execução de testes automatizados} – Testes unitários e de integração garantem que mudanças não causem regressões.  \newline
3. \textbf{Relatórios de qualidade} – Ferramentas como SonarQube e Prometheus podem ser integradas para atestar a qualidade do código e verificar se há problemas na sua execução.  \newline
4. \textbf{Feedback rápido} – Se um erro for detectado, os desenvolvedores recebem notificações imediatas para corrigir antes da fusão com a branch principal.  \newline

Utilizando essa abordagem, problemas de compatibilidade são identificados com rapidez, reduzindo retrabalho e garantindo que o código esteja sempre em um estado funcional antes de ser integrado ao sistema principal.  

\subsubsection*{Caso 2: Problemas com deploy manual e falhas frequentes}  
A startup do exemplo enfrenta dificuldades no processo de deploy manual, pois esse método é defasado e não garante um funcionamento uniforme do processo de deploy. Utilizando CD esse problema pode ser resolvido automatizando a preparação e implantação das atualizações, garantindo um fluxo mais confiável e eficiente.  

Com o CD, todas as mudanças que passam pela \textbf{Integração Contínua} são automaticamente preparadas para implantação, eliminando o risco de erros humanos no deploy. Para essa startup, uma pipeline automatizada poderia seguir as seguintes etapas:  

1. \textbf{Build e Testes Automatizados} – A cada novo \textit{commit}, o código é compilado e testado para garantir que está funcionando corretamente.  
2. \textbf{Criação dos pacotes} – O software é empacotado (por exemplo, um container \textit{Docker}) para garantir consistência entre os ambientes.  
3. \textbf{Implantação Automatizada} – O código é enviado para os servidores de produção automaticamente, eliminando a necessidade de deploy manual. Como por exemplo utilizando o \textbf{GitLab CI/CD}.  
4. \textbf{Monitoramento e Rollback} – Após o deploy, ferramentas como \textit{Zabbix, Datadog, Grafana, Graylog e ELK Stack} podem monitorar o sistema em tempo real. Caso algum problema seja detectado, um rollback automático pode ser acionado.  

Com a \textbf{Entrega Contínua}, a startup pode manter seu ciclo de lançamentos semanais sem instabilidades, garantindo que novas funcionalidades sejam entregues rapidamente e com menos chanbce de conter erros.  


\newpage{}

\section*{Conclusão}
\lipsum[4] 

\end{document}

