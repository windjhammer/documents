\documentclass[a4paper,12pt]{article}
\usepackage[utf8]{inputenc}
\usepackage[brazil]{babel}
\usepackage{enumitem}
\usepackage{hyperref}

\title{5 Produtos Oferecidos pela AWS e suas Funcionalidades}
\author{Breno Fernandes - 01169313}
\date{\today}

\begin{document}

\maketitle

A Amazon Web Services oferece uma vasta gama de serviços baseados em nuvem, permitindo que empresas de todos os portes hospedem seus serviços e os escalem de acordo com suas necessidades.

\section*{Amazon EC2 - Elastic Compute Cloud}
\begin{itemize}[left=0pt]
    \item \textbf{O que é:} É o serviço de computação em nuvem mais popular da AWS. Ele permite a criação de instâncias para um servidor virtual.
    \item \textbf{Para que serve:} Ideal para hospedar aplicações web, executar aplicações de big data, realizar testes e POCs. É possível escolher entre uma variedade de tipos de instância, sistemas operacionais e configurações para atender às mais variadas necessidades.
\end{itemize}

\section*{Amazon S3 - Simple Storage Service}
\begin{itemize}[left=0pt]
    \item \textbf{O que é:} Um serviço de armazenamento de objetos focado em escalabilidade e durabilidade.
    \item \textbf{Para que serve:} Perfeito para armazenar dados não estruturados, como arquivos, backups, imagens e vídeos. O S3 é utilizado em diversas aplicações, desde sites estáticos até arquivos de log e bancos de dados de objetos.
\end{itemize}

\section*{Amazon RDS - Relational Database Service}
\begin{itemize}[left=0pt]
    \item \textbf{O que é:} Um serviço gerenciado de bancos de dados relacionais que oferece suporte a diversos motores de banco de dados populares, como MySQL, PostgreSQL e Oracle.
    \item \textbf{Para que serve:} Simplifica a configuração, operação e dimensionamento de bancos de dados relacionais na nuvem. É ideal para aplicações que exigem um banco de dados relacional, como aplicações web tradicionais e CRM.
\end{itemize}

\section*{Amazon Lambda}
\begin{itemize}[left=0pt]
    \item \textbf{O que é:} Um serviço de computação serverless que permite executar código sem provisionar ou gerenciar servidores.
    \item \textbf{Para que serve:} Ideal para executar código em resposta a eventos, como uploads de arquivos, chamadas de API ou mudanças de dados. É amplamente utilizado para construir aplicativos back-end, processar dados em tempo real e criar APIs.
\end{itemize}

\section*{Amazon CloudFront}
\begin{itemize}[left=0pt]
    \item \textbf{O que é:} Um serviço de CDN globalmente distribuído que acelera a entrega de conteúdo estático e dinâmico para seus usuários finais.
    \item \textbf{Para que serve:} Melhora a performance dos seus sites e aplicativos, reduzindo a latência e aumentando a disponibilidade. É ideal para entregar conteúdo como vídeos, imagens, arquivos e APIs para usuários em todo o mundo.
\end{itemize}

\section*{Precificação dos Serviços AWS}
A AWS utiliza um modelo de precificação por uso, o que significa que o usuário paga apenas pelos recursos que utiliza. Cada serviço da AWS tem sua própria estrutura de preços, que pode variar de acordo com o tipo de recurso, a quantidade utilizada e a região geográfica.

\subsection*{Principais fatores que influenciam a precificação:}
\begin{itemize}[left=0pt]
    \item Tipo de instância: O tamanho, a capacidade de processamento e a memória da instância EC2 influenciam o preço.
    \item Armazenamento: A quantidade de dados armazenados no S3, o tipo de armazenamento (Standard, Infrequent Access, Glacier) e a classe de armazenamento (S3 Standard, S3 Intelligent-Tiering, S3 Glacier Instant Retrieval) influenciam o preço.
    \item Transferência de dados: A quantidade de dados transferidos entre a sua instância e a internet, ou entre regiões, influencia o preço.
    \item Uso de outros serviços: O uso de outros serviços da AWS, como RDS, Lambda e CloudFront, também gera custos adicionais.
\end{itemize}

\subsection*{Características do modelo de precificação da AWS:}
\begin{itemize}[left=0pt]
    \item Pagamento por uso: O usuário paga apenas pelos recursos que utiliza, o que oferece flexibilidade e escalabilidade.
    \item Transparência: A AWS disponibiliza calculadoras de custos detalhadas para que o usuário possa estimar os custos antes de iniciar o uso dos serviços.
    \item Descontos: A AWS oferece descontos por utilização contínua, reservas de instâncias e compras em grandes volumes.
\end{itemize}

\end{document}

